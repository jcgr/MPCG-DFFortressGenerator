\section{Methods}
\label{04}
There are many different algorithms for dungeon generation\cite[Chapter 3]{PCGBook}, but none that fit our needs specifically. Because we want to generate map layouts where all rooms are connected to the entrance, but not necessarily each other, agent-based dungeon growing did not feel like the right choice. Instead we made the tool using our own map generation algorithm.

The map generation algorithm consists of two different parts: A layout generation algorithm to generate the basic layout of the map (see section \ref{04_MapGeneration}) and evolutionary algorithms\cite[Chapter 2]{IoEC} to determine how the map layout can be used in the best way (see section \ref{04_Evolution}).

\subsection{Interface}
\label{04_Interface}
The interface was created for two purposes: To let the user select options for the map generation easily and to let the user browse the generated maps. 

We accomplished this by splitting the interface in two parts. The right part (see figure \ref{fig:ToolRight_02}) contains all the options for the map generation. The user can select the dimensions of the map surrounding the fortress to be generated, the number of dwarves that should inhabit the fortress and which rooms that are required to have in the fortress.

The left part (see figure \ref{fig:ToolLeft_02}) shows an empty box until maps have been generated, at which point it shows the map the user has selected (by default it selects the map with highest fitness). There are two drop-down menus which allows the user to select which of the generated maps they want to see (ordered by descending fitness values) and what layer of the map they are shown.

\insertPicture{0.5}{ToolLeft_02}{}

\insertPicture{0.2}{ToolRight_02}{}

% Map generation algorithm
\subsection{Map generation}
\label{04_MapGeneration}
Our layout generation algorithm is as follows:
\begin{enumerate}

	\item Create a map of the user-specified dimensions and calculate the number of rooms we want, based on how many dwarves the user expects to have in their fortress.

	\item Create an entrance to the fortress on the top layer along one side of the room.

	\item For each layer in the map (starting at the top one), make a list that contains all positions that are not dug out and then do the following:

	\begin{enumerate}
		
		\item Choose a random position from the list of open positions and use it as one corner of the room.

		\item Pick a random, diagonal direction (north/east, south/east, south/west or north/west), go 6 tiles that way and see if the positions in that direction are open.

		\item If they are open, build an empty room in the middle 4x4 square, set the tiles around that square to room walls, and remove the positions now occupied by the room from the list of open positions.

		\item Create a path from the room to the nearest entrance using Breadth-first search\cite{sedgewick4th}\cite{ucibfs}(stairs on the lower levels count as entrances to that level).

		\item Once enough rooms have been created, or only 10\% of the layer is open, create stairs to the next layer, and connect these stairs to the entrance of the current layer.

		\item Repeat steps a) to e) for each layer until enough rooms have been built.

	\end{enumerate}
\end{enumerate}

If the algorithm fails to generate a staircase (this is most common when selecting small maps with many dwarves), it will stop generating more rooms and begin doing evolution.

\subsection{Evolution}
\label{04_Evolution}
While evolutionary algorithms work on the same principles, there are some points we feel are important to discuss in our implementation.

\subsubsection{Initialization of maps}
\label{04_Evolution_Initialization}
In order to evolve maps we need maps. Prior to the evolution, we generate 10 maps using the map layout algorithm described in section \ref{04_MapGeneration}. For every map we generate, we also find and save the distance from every room to every other room, as we need the distance for our objective function. Saving the distances ahead of time allows us to save time when we need the distances as we can simply look them up.

We find the distances between rooms by using breadth-first search\cite{sedgewick4th}\cite{ucibfs}. Every room finds and saves the distance it has to every other room. In order to save time during this step, we save the direction in both directions when a distance is found (the start room knows the distance to the target room and the target room knows the distance to the start room). Other ways of finding the distances are discussed in \ref{06_Distances}.

\subsubsection{Candidates for evolution}
The genotypes (the representations used for evolution) are not the maps that are generated, but rather what is in each room in the map. The problem of evolving optimal map layouts is two-fold: one search space is the placement of each room, e.g. room 01 is located at location (22,13), and the other search space is the combination of room types ("room assignments") in the layout, e.g. room 01 is a barracks. Our algorithm only mutates room assignments and not room placement, as optimizing in both search space is very expensive. 

This means that each map has its own set of room assignments that are not mixed between maps. The evolution runs for 100 generations. Every generation of the evolution the current best room assignment (parent) of each map spawns 100 room assignments (children) that are mutations of the parent. Every room in a child has a 30\% chance to be changed into a random room. The child with the best fitness value of these 100 children is then compared to its parent. If the fitness value of the child is better than or equal to the fitness value of the parent, the child replaces the parent as the current best room assignment for that particular map.  

Using this method of evolution, we essentially select a random subset of the "room placement" search space and search the "room assignment" search space of each element in the subset of "room placement". By never changing the original random select of room placements, we ensure that we do not converge towards a local optima too quickly.

\subsubsection{Objective function}
Our objective function works in two steps. First we check if the room assignment being evaluated has all the different types of rooms that the user requested. For each missing room, the fitness is penalized depending on how many generations have already passed (later generations are penalized more heavily).

After that, it calculates the the fitness value of each individual room. This calculation is based on the distance between the room and other rooms it has a relation to. Depending on whether it should be far from, or close to, the other rooms, its fitness is changed. 

The relation between rooms is something we have defined based on how we felt rooms related to each other in the game. As an example, it makes sense for a barracks to be near the entrance, as it allows the military to respond to any threat quickly, while the bedrooms should be far from the entrance for exact opposite reason; having monsters was barge in on sleeping dwarves is never a good thing.

When relations to other rooms have finished, the function also checks how many we have of that type of room already. If we have more than we expect we will need (varies depending on room type), the room's fitness is lowered to a factor of $1 / 2^{numberOfExcessRooms}$. After this is done, the value of the room is added to the room assignment's total fitness value and the the process starts over with the next room.

\subsubsection{The algorithm as pseudo-code}
\begin{enumerate}

	\item Create 10 different maps layout (using the map generation algorithm, see below) with empty rooms. When a map has been generated, find the distance from any room to all other rooms (see \ref{04_Evolution_Initialization}).

	\item For each map layout, create create a random initial room assignment.

	\item For every map layout, do the following every generation until enough generations have passed:

	\begin{enumerate}

		\item Use the current best room assignment as candidate for the generation.

		\item Create 100 mutations of the candidate.

		\item Calculate the fitness value for each room assignment.

		\item If a room assignment is better than its parent, replace the parent with the room assignment.

	\end{enumerate}

	\item At the end, for every map, copy its best room assignment into the actual map.

\end{enumerate}

\subsection{Taxonomy}
\label{04_Taxonomy}

Following the taxonomy in \cite{togelius2010search}, we have classified our algorithm as follows:

\begin{itemize}

	\item \textbf{Offline} - Our algorithm is run as part of a tool that does not integrate into Dwarf Fortress\footnote{ It is also quite slow which would make it poor for an online tool should it later be interfaced with the game\cite{togelius2011search}.}.

	\item \textbf{Optional Content} - It is up to the user whether they want to use the tool (and thereby the algorithm) or not.

	\item \textbf{Parameter Vectors} - The algorithm takes some parameters from the user (fortress dimensions, number of dwarves and which rooms the user wants).

	\item \textbf{Stochastic Generation} - Most of the generation uses randomness in order to determine what to do.

	\item \textbf{Generate-and-test} - We use evolutionary programming as part of the algorithm, which, by definition, tests everything it generates\footnote{It does, however, stop after a certain number of iterations and returns the best results.}.

\end{itemize}

%Evolution algorithm

% Objective function: Should factor in that there are some rooms that we do not want a lot of.

%How does your algorithm work? Describe in as much detail as you can fit into the report. Also, how did you interface it to the game?

%Algorithm: Generate various layouts by digging rooms in random (open) positions, then connecting them to the entrance. Evolve/mutate by swapping room content around for each layout and evaluating each room assignment for every layout.

%Discuss Dijkstra + AStar, fitness function, how distances are calculated (time saved later by spending time early)

%PCG Taxonomy: classify

%Multi-threading:

%Interface: We did not interface it directly into the game. We give the user a layout map that provides the best room layouts generated.

% Distance improvements:
% Find distance to rooms while generating path to entrances.
% For distance between multiple layers: Save distance to entrance/exit for every room and just add them together.
% Dijkstra: Use Dikjstra to find distance to ALL rooms instead of just looking for one specific room at a time.