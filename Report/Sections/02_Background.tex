\section{Background and Game Design}
\label{02}
Procedual content generation (PCG) has been around for a long time, but it continues to evolve and improve as more reason is done in the area. It is used to create content for a game through algorithms. The main benefit of PCG is that it often produces content faster and cheaper than manually creating content.

PCG can also be used to help players. For example, a PCG tool for Minecraft would be able to suggest different ways of building a base to the player, either from the ground or based on something the player had already built. Most PCG tools are made for developers to use in an offline context or as part of the game itself in order to cut down costs. Most games are completely fine without PCG tools for player use, but there are a few games out there where it would benefit the player greatly if there were PCG tools to help in decision making. One such game is Dwarf Fortress.

\subsection{Dwarf Fortress}
\label{02_DF}
In Dwarf Fortress the player leads an expedition of dwarves in order to create a new home for them. The game takes place in a 3D map (displayed to the player as a 2D map with multiple layers), where the player can direct dwarves to perform various tasks (mining, gathering plants, building furniture, crafting weapons, and so forth). The goal of the game is to build a huge fortress for the dwarves and to keep them alive for as long as possible.

Dwarf Fortress is often described as having a very steep learning curve, as it tells the player nothing about how to play or what they are supposed to do. All of it is something the player has to figure out by themselves. This often means that a player's first fortresses will be of very low quality, as they discover more and more things they have to add to it that they did not plan for.

\insertPicture{0.5}{DFScreenshot_01}{A screenshot of a very basic fortress.}

%In order to help the novice player out, we created a fortress generator that can give them some basic layouts from which they can choose which one they like the most. These layouts also give the player an idea of how many different things they need to play for in any future fortresses they may want to make.

\subsubsection{Reducing the problem}
\label{02_DF_Reducing}
Dwarf Fortress contains an immense amount of possibilities. There are 31 workshops\cite{dfwiki:Workshops} from which dwarves can craft different objects, 12 different types of rooms\cite{dfwiki:Rooms} each with their own function and 16 different stockpiles\cite{dfwiki:Stockpiles} which are used for storing specific types of items. These are all related in different ways and even with just the elements mentioned above, it is already overwhelming for most new players. That is without mentioning special types of constructions, mechanics and the militaristic area of the game.

In order to preserve clarity in the tool, we reduced the amount of possibilities we include. Instead of including all types of rooms, workshops and stockpiles (from now on referred to simply as "rooms"), we cut it down to the ones we felt were the most essential ones to any fortress. This reduced the total number of rooms from 59 to 23, a much more manageable number when making the interface.

Due to the nature of the algorithm, it is entirely possible to expand the number of room types without significant change in run time. It would simply make the interface of the pool more cluttered.

\subsection{Evolutionary programming}
\label{02_Evolution}
Evolutionary programming\cite[Chapter 2]{IoEC} is based on biological evolution. It works on the principle that if the program knows how to evaluate the quality (fitness) of an object and it knows how to create a variation of this object, over time it will be able to search the space of possible solutions for a problem and chose an optimal solution (either a local or global optimum).

Evolutionary programming is often used for optimization problems, as they are quite solid in what results they produce. Assuming that the evaluation function is representative of how an object should be evaluated and the variation operators create the necessary diversity (and thus facilitate novelty), the fitness of a population should continuously improve as the algorithm attempts to imitate real world evolution.

This also applies to fortresses in Dwarf Fortress. As creating a good fortress layout is, in essence, an optimization problem, evolutionary programming is a very suitable technique to use. Compared to doing it "by hand", evolutionary programming is able to explore far more possibilities in a much shorter time frame and still arrive at a decent, if not great, result in the end.