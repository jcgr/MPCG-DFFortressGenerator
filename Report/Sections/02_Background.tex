\section{Background}
\label{02}

Procedual content generation has been around for a long time, but it continues to evolve and become better. As it is now, it is mostly used by game creators to create content for their games. There are, however, some games where a PCG tool will be able to help the players learn more about the game and how to do better while playing.

\textbf{Stuff about pcg/evolution}

\subsection{Dwarf Fortress}

Dwarf Fortress is a game about leading an expedition of dwarves in order to create a new home for them. The game takes place on a procedurally generated map with multiple layers, where the player can direct dwarves to perform various tasks (mining, gathering plants, building furniture, crafting weapons, and so forth). The goal of the game is to build a huge fortress for the dwarves and to keep them alive for as long as possible.

Dwarf Fortress is often described as having a very steep learning curve, as it tells the player nothing about how they play or what they are supposed to do. All of it is something the player has to figure out by themself. This often means that a player's first fortresses will be of very low quality, as they discover more and more things they have to add to it that they did not plan for.

In order to help the novice player out, we intend to create a fortress generator that can give them some basic layouts from which they can choose which one they like the most. These layouts should also give the player an idea of how many different things they need to play for in any future fortresses they may want to make.

\insertPicture{0.5}{DFScreenshot_01}{A screenshot of a very basic fortress.}

\subsection{Evolutionary computing}
\label{02_Evolution}

Evolutionary computing\cite[Chapter 2]{IoEC} is based on biological evolution. It works on the principle that if the program knows how to evaluate if an object is good and it knows how to change this object in order to affect how good/bad it is, then it should be able to create a good result given enough time.

Evolutionary computing (EC) is often used for optimization problems, as they are quite solid in what results they produce. Assuming that the evaluation function is well written, the evolution should keep moving towards better results, without the need for humans to constantly change things.

This also applies to fortresses in Dwarf Fortress. As creating a good fortress layout is, in essence, an optimization problem, evolutionary computing is a very suitable technique to use, as it is able to explore a lot of possibilities that would not be feasible by "hand".



%Has this been done before? How? If not, what’s the closest related research? (Both using similar approaches and other algorithms.) What’s novel with your research?

%Dungeon generators have been made before, but not for this specific game. Minecraft? Generation of 3d maps

%Novelty? /shrug