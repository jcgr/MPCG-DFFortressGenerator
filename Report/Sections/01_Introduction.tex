\section{Introduction}
\label{01}
This paper was written as part of a project for the "Procedural Content Generation in Games" course on the Games Technology track at the IT-University of Copenhagen. Both the source code\footnote{https://github.com/jcgr/MPCG-DFFortressGenerator} and test data\footnote{https://drive.google.com/file/d/0B-DyxsuStO29cUNteHRKN055a2c /view?usp=sharing} for the project is available online.

Games that involve planning the layouts of buildings, rooms, cities, etc. often require the user to plan far ahead in order to create the perfect layout. The complexity of these games can often be so great that new users are put off by the steep learning curve and decide not to play at all. One such game is Dwarf Fortress\cite{DwarfFortress}. Dwarf Fortress focuses on building a large, underground fortress, where dwarves work, eat, and sleep. In order for the dwarves to be effective, the player must create a layout that facilitates effectiveness, e.g. workshops being located close to stockpiles that contain the materials they need. Procedural Content Generation can be used to ease processes such as these and the intent of this paper is to showcase an algorithm that can be used to generate it layout for fortresses in Dwarf Fortress.

The aim of this paper is to generate useful fortress layouts for Dwarf Fortress using evolutionary programming. We begin by giving a short overview of Dwarf Fortress and Evolutionary programming. Then we introduce the method we use to generate layouts and how we evolve the layouts in the search for an optimal layout. After presenting our method, we classify our algorithm using the Procedural Content Generation taxonomy\cite{togelius2010search}. A report of our results follows the classification and we round out the paper with a discussion of alternative methods and potential future work.