\section{Introduction}
\label{01}

This paper was written as part of a project for the "Procedural Content Generation in Games" course on the Games Technology track at the IT-University of Copenhagen.

For the project we could do  one of two things: Develop a new PCG algorithm, or apply an existing algorithm (or combination of algorithms) to some game domain in a novel way.

\subsection{Problem Statement}
\label{01_01}

For our project, we decided to generate fortresses for the game Dwarf Fortress\footnote{http://www.bay12games.com/dwarves/} by creating random map layouts and and then evolve the contents of the maps.

Our goal was to be able to generate fortresses that could be used in-game in practice, even though they may not be perfect fortresses\footnote{"Perfect" in this context meaning a fortress that encapsulates everything a fortress can have, with a layout that optimizes everything.}. We decided to not aim for perfect fortresses, as Dwarf Fortress is an amazing complex game and there are an immense number of variables to take into account in order to create a perfect fortress.

%What problem are you trying to solve? Why is this important? How does this problem make some sort of game better?

%Dwarf Fortress is complex and we want to help the user figure out a good layout for their fortress.