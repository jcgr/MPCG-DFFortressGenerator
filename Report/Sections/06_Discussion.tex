\section{Discussion}
\label{06}
There are three main points of discussion on the choices we made with regards to our method: choice of algorithm and method for calculating distances, our objective function, and our mutation method. It is also of interest how the method could potentially be used in generating layouts for fortresses and similar buildings in other game genres.

\subsection{Distance Calculations}
\label{06_Distances}
As discussed in section \ref{05}, the dominating factor on time spent when evolving maps is the calculation of distances between rooms. Because the objective function requires that the distance from every room to every other room is known, these distances must be calculated at some point during the generation of the maps. Our use of breadth-first search in the method described in section \ref{04} is quite ineffective and has potential for vast improvements.

\subsubsection{Choice of Algorithm}
Because we currently calculate distances individually, the use of a heuristic in our search would highly beneficial. As straight line distance (SLD) would be clearly admissible (no path can be shorter than SLD), we could benefit from the use of A*\cite{AStar, AStarOriginal} using SLD as the heuristic. Since A*'s performance is O(E)\cite{russell3rd}\footnote{Assuming optimal heuristic.} where E is the number of edges, this would reduce the time complexity of the distance calculations to O($N^2(4V)$).

\subsubsection{Calculation of Distances}
While A* would certainly be more effective with our current approach, a different optimization would be to continue using breadth-first search. If we searched for the distance to all other rooms at once instead of individually, we could reduce the current O($N^2(4V+V)$) time complexity to O($N(4V+V)$), which is clearly desirable. 

\subsubsection{Separate Layers}
A further improvement would be to only calculate the distance to the entrance and exits of a layer. Whenever distance calculation requires finding the distance to a room on a different layer, it is simple to find the distance to the entrance/exit and add the distance from that tile to the target tile. This would mean that N will be equal to the number of rooms on the layer instead of the total number of rooms in the entire map (significantly speeding up the process when a large number of rooms is required).

\subsection{Alternative Objective Function}
\label{06_AlternateObjective}
An alternative to our current objective function would be to reward novelty. With the current objective function, we encourage sameness as we are unlikely to have a high fitness value if an area contains many areas with no correlation. Adding a bonus to the fitness score of a layout dependent on the number of unique rooms it contains would encourage novelty in area assignments.

\subsection{Alternative Mutation Method}
\label{06_AlternateMutation}
Our current mutation method is not very vulnerable to local optima until later on in the generation due to the missing room penalty increasing with each generation. In order to speed up map generation, however, it might be desirable to hit a local optimum earlier. This could be done by mutating area assignments as we do currently, but in addition creating a copy of the assignment with a random ordering, then comparing the two and using the one with the better fitness value when comparing to the parent. This would test two area assignments per mutation, which would then allow for a reduction in the number of generations done in evolution. However, fewer generations would means fewer alternatives being tested, which could potentially result in the evolution getting stuck in a local optimum.

\subsection{Generalization To Other Game Genres}
\label{06_Generalization}
While the method is not likely to be useful in an online context, as it is simple too slow (online PCG must be fast\cite{togelius2011search}), it could be useful for suggesting a layout of a castle, fortress or other similar structure to a level designer looking for inspiration. Such a change would require a different interface that could take into account the needs for specific types of rooms (and the weights of their relation), but the method itself would still be fully functional.

\subsection{Future work}
Besides the improvements mentioned in the discussion above, there are a number of smaller things we would consider improving should we do any future development on the tool.

We currently assume that all rooms will be four by four tiles in size. This is rarely the case in actual play, so changing the size of a generated room depending on the room type would be improve how well the output of the tool represents the actual game. It does, however, introduce the problem of how to generate the map layouts with a variable room size. This could potentially be solved using a lower and an upper-bound on room sizes.

The method for placing rooms described in section \ref{04} concentrates the placed rooms on the upper layers of the map. This is not always desirable, as the lower levels of the map will be unused. It also leaves little room for expansion on the upper layers. A potential solution to this problem would be to place an equal percentage of rooms on each layer selected by the user. This would ensure a somewhat even distribution of rooms on all layers and would reduce concentration of rooms. This could lead to longer paths in general, but this should not result in worse performance if the distance calculations is separated into layers as suggested in section \ref{06_Distances}.

Because the tool is currently not able to show more than 50 tiles on the vertical axis, some of the map is cut out when generating large maps. This is impractical for the user, so adding a scroll function (or similar) to the map window would be a definite improvement to the tool.

In addition, the map is currently displayed in ASCII characters. This is, at best, not very readable as the characters are taller than they are wide. Changing the ASCII characters into the characters used by the real game or to another graphical representation would improve the usability of the tool.

Adding a save function to the layout generation tool could be beneficial to a user, as it is currently necessary to record the maps, if they must be saved for later use, e.g. by taking a screenshot of each layer.

\subsection{Contribution to the project}
Most of the work done on the project was shared equally between both of the authors, but there were some areas which were primarily done by one author.

Generation of initial map layouts and the user interface was primarily programmed by Jacob Grooss. Evolution of the maps and the distance calculations were primarily programmed by Jakob Melnyk.

Sections \ref{02}, \ref{04}, and \ref{05} of the paper was primarily authored by Jacob Grooss.  Sections \ref{01} and \ref{06} as well as the abstract was primarily authored by Jakob Melnyk.