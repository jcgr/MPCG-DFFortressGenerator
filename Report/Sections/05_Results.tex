\section{Results}
\label{05}

In our tool, there are five parameters that can be adjusted (internally, with no interface to the user) in order to influence the evolution: 

\begin{itemize}

	\item \textbf{Mutation chance} - The chance that a room is transformed into a random type of room during evolution.

	\item \textbf{Number of children} - The number of children that are spawned for each map every generation.

	\item \textbf{Number of generations} - The amount of generations to evolve the maps over.

	\item \textbf{Missing room penalty} - How much a room assignment is penalized for every required room it does not contain.

	\item \textbf{Missing room penalty scaling factor} - How much the missing room penalty is scaled up for every generation that has passed. The idea is that the closer we get to the end of the evolution, the more important it is to make sure a room assignment had the required rooms.

\end{itemize}

There were two things we wanted to test in regards to the maps our tool generates:

\begin{enumerate}

	\item Are the better rated maps actually better than the others? (I.e. does our fitness function work as we would expect it it?)

	\item What influence, if any, does changing the parameters have on the quality of the generated maps?

\end{enumerate}

In order to answer both questions, we set up the tool so it could run evolutions with different values for the parameters. We created a combination of each of the following values for the parameters, where the number of generations was equal to $\frac{10.000}{numberOfChildren}$:

\begin{center}
	\begin{tabular}{| l | c | c | c | }
		\hline
	  	Mutation chance 			& 0.1 & 0.2 & 0.3 \\ \hline
	  	Number of children  		& 10 & 100 & 1000 \\ \hline
	 	Missing room penalty 		& 10 & 100 & 200 \\ \hline
	  	Missing room scaling factor 	& 10 & 100 & 200 \\ \hline
	\end{tabular}
\end{center}

With the number of generations being either 1000, 100 or 10, we had a total of 81 different parameter value combination. 

In order to test all combinations equally, we generated 50 maps at the beginning. For each combination, we cleared all 50 maps of their content (meaning that all rooms were empty), loaded the values for the parameters from the combination and then ran the evolution on every one of the maps. At the end of each combination, we saved the generated maps - along with their fitness values - to a unique text file, which we could then examine.

The tests were run on a map with the dimensions 40x40x4 with room for 30 dwarves. We felt that 30 dwarves was a good number, as it is a manageable amount for most players once they have grasped the basics of the game and was within a reasonable time frame for us to test.

The rooms chosen as required rooms were bedrooms, dining rooms, barracks, farms and officies. From workshops we chose the carpenter's, the craftdwarf's, the kitchen and the mason's workshop. From the stockpiles we chose finished goods, food, furniture, stone and wood. All of these are what we feel make up the very essential of any basic fortress.

\subsection{Quality of Maps}

\textit{Question 1} required us to look through some of the samples and manually compare the best rooms to the other rooms. We compared the best map to the worst map and to the map that was evaluated to be average. Comparing to more maps could provide more precise results, but we felt the trade-off in time was not worth it. This is particular relevant as our manual evaluation metric is subjective, even if this subjectivity is influenced by experience from playing the game.

We looked at 10 of the 81 datasets, all randomly chosen. For all 10, the map that was rated as the best was better than the two we compared it to. Had it not been the case for one or more, we would have looked at more datasets. As we looked at more than 10\% of the different datasets and the data confirmed that better maps had higher fitness than worse maps, we believe our fitness function to be effective. Due to the subjective nature of our evaluation (and our objective function), it is possible that the evaluation of maps is not extremely precise. This potential lack of precision is not too big a worry, however, as most users of this type of tool are not likely to use it with the intent for perfection; instead it is a guideline to a layout that can be expanded through gameplay.

In order to answer \textit{Question 2}, we compared the different parameters by keeping them static and comparing the best maps generated where that parameter was the same. For map layouts we used the map layouts for all tests, such that the randomly selected subset of room placements did not influence the change in parameters.

What we learned was surprising. The mutation chance did not have any notable influence on the generated maps. We believe this is due to the fact that we can spawn so many children during evolution (10.000 in total for each room placement), that even with a low mutation chance there is enough time to "catch up" to the variety in children that the higher mutation chances have.

The number of children and number of generations also did not have any notable influence, which is due to the fact that we chose numbers where $numberOfChildren \times numberOfGenerations = 10.000$. Had we chosen test set-ups where the multiplicative of these two parameters did not equal 10.000, it is likely that we would have seen a difference in the quality of the maps.

Only the missing room penalty and its scaling factor had any real influence on the quality of the generated maps. With the values $missingRoomPenalty = 100$ and $missingRoomPenaltyScalingFactor = 10$, most of the required rooms were present in any map. At 200 and 100, respectively, all rooms were present in all of the best maps.

\subsection{Speed}
During testing, it became clear that we experienced significant slowdown in generation of map layouts when we added more dwarves.Testing revealed that approximately 90\% of the total run time of our algorithm was spent calculating the distances between rooms.



During testing, we noticed that the program ran slower the more dwarves we wanted to generate maps for. To figure out if it was a specific part of the program that was the problem, or if the entire program simply was slow, we timed the different parts of the evolution algorithm.

The reason for this is that we find the distances in a very non-efficient manner. Every room finds the distance to every other room, which has a cost of $N^2$ where N is the number of rooms. While we cut that down to $N(N+1)/2$ by saving the distance both ways when a target was found, it is still $O(N^2)$.

The problem is that we use Breadth-First Search\cite{sedgewick4th}\cite{ucibfs} every time we find the distance between rooms. With the way our map works, BFS has a worst case time complexity of $O(4V + V)$ where V is the number of tiles on the map. $4V$ comes from the fact that every tile will check its 4 neighbour tiles to see if they have been visited or not and worst case every tile will have to do so. This leads to a total cost of $O(N^2 (4V+V))$, which is an immense cost investment compared to our other functions.

The subject of how to speed up the algorithm is discussed in section \ref{06_Distances}.


%Did it work? How well? Provide some figures, and a table or two. How much time does it take? Remember to include significance values (remember the t-test?), variance bars… Reread some of the papers from class and compare how they report their results.

%PCG Book 2.4: Evaluation functions

%PCG Book 12: Evaluating Generators

%Time taken (discuss number of evaluations and refer to method chapter)